\documentclass{thesisclass}
% Based on thesisclass.cls of Timo Rohrberg, 2009
% ----------------------------------------------------------------
% Thesis - Main document
% ----------------------------------------------------------------
% No empty in-between chapters, remove this line if desired for double-page printing
\let\cleardoublepage\clearpage

%% -------------------------------
%% |  Information for PDF file   |
%% -------------------------------
\hypersetup{
 pdfauthor={Not set},
 pdftitle={Not set},
 pdfsubject={Not set},
 pdfkeywords={Not set}
}


%% ---------------------------------
%% | Information about the thesis  |
%% ---------------------------------

\newcommand{\myname}{Name}
\newcommand{\mytitle}{Titel der Ausarbeitung bzw. des Themas}
\newcommand{\myinstitute}{Institut für Visualisierung und Datenanalyse,\\ Lehrstuhl für Computergrafik}

\newcommand{\reporttype}{Master}                                  % Proseminar-/Seminar-/Bachelor-/Master
\newcommand{\reviewerone}{Prof. Dr.-Ing. Corsten Dachsbacher}     % only for bachelor/master
\newcommand{\reviewertwo}{Prof. Dr. Hartmut Prautzsch}            % only for bachelor/master
\newcommand{\advisorone}{?}
%\newcommand{\advisortwo}{?}                                      % only if applicable

\newcommand{\timestart}{XX. Monat 20XX}
\newcommand{\timeend}{XX. Monat 20XX}
\newcommand{\submissiontime}{DD. MM. 20XX}


%% ---------------------------------
%% | ToDo Marker - only for draft! |
%% ---------------------------------
% Remove this section for final version!
\setlength{\marginparwidth}{20mm}

\newcommand{\margtodo}
{\marginpar{\textbf{\textcolor{red}{ToDo}}}{}}

\newcommand{\todo}[1]
{{\textbf{\textcolor{red}{(\margtodo{}#1)}}}{}}


%% --------------------------------
%% | Old Marker - only for draft! |
%% --------------------------------
% Remove this section for final version!
\newenvironment{deprecated}
{\begin{color}{gray}}
{\end{color}}


%% --------------------------------
%% | Settings for word separation |
%% --------------------------------
% Help for separation:
% In german package the following hints are additionally available:
% "- = Additional separation
% "| = Suppress ligation and possible separation (e.g. Schaf"|fell)
% "~ = Hyphenation without separation (e.g. bergauf und "~ab)
% "= = Hyphenation with separation before and after
% "" = Separation without a hyphenation (e.g. und/""oder)

% Describe separation hints here:
\hyphenation{
% Pro-to-koll-in-stan-zen
% Ma-na-ge-ment  Netz-werk-ele-men-ten
% Netz-werk Netz-werk-re-ser-vie-rung
% Netz-werk-adap-ter Fein-ju-stier-ung
% Da-ten-strom-spe-zi-fi-ka-tion Pa-ket-rumpf
% Kon-troll-in-stanz
}


%% ------------------------
%% |    Including files   |
%% ------------------------
% Only files listed here will be included!
% Userful command for partially translating the document (for bug-fixing e.g.)
\includeonly{%
titlepage,
content
}


%%%%%%%%%%%%%%%%%%%%%%%%%%%%%%%%%
%% Here, main documents begins %%
%%%%%%%%%%%%%%%%%%%%%%%%%%%%%%%%%
\begin{document}

% Remove the following line for German text
%\selectlanguage{ngerman}
%\selectlanguage{english}

\frontmatter
\pagenumbering{roman}
\include{titlepage}
\blankpage


%% -------------------
%% |   Directories   |
%% -------------------
\tableofcontents
\blankpage


%% -----------------
%% |   Main part   |
%% -----------------
\mainmatter
\pagenumbering{arabic}
%% content.tex
%%

%% ==============================
\chapter{Einleitung}
\label{ch:Introduction}
%% ==============================

\section{Erster Abschnitt}
Hier beginnt der Text...\\
%
Und so sieht eine Referenz aus \cite{becker2008a}!\\[3em]

%
Und so ein Bild:\\
\begin{figure}[h]
  \begin{center}
    \includegraphics[width=.3\textwidth]{logos/KITLogo_RGB.pdf}
    \caption{Das ist eine Bildunterschrift}
  \end{center}
\end{figure}

\section{Zweiter Abschnitt}

Dies ist ein langer Text, der dafür sorgt, dass alsbald ein Zeilenumbruch erfolgt: $x$"~Koordinatensystem. Lorem ipsum dolor sit amet, consetetur sadipscing elitr, sed diam nonumy eirmod tempor invidunt ut labore et dolore magna aliquyam erat, sed diam voluptua. At vero eos et accusam et justo duo dolores et ea rebum. Stet clita kasd gubergren, no sea takimata sanctus est Lorem ipsum dolor sit amet. Lorem ipsum dolor sit amet, consetetur sadipscing elitr, sed diam nonumy eirmod tempor invidunt ut labore et dolore magna aliquyam erat, sed diam voluptua. At vero eos et accusam et justo duo dolores et ea rebum. Stet clita kasd gubergren, no sea takimata sanctus est Lorem ipsum dolor sit amet.



%% ==============
\chapter{Ein Kapitel}
\label{ch:Content1}
%% ==============


%% ===========================
\section{Erster Abschnitt}
\label{ch:Content1:sec:Section1}
%% ===========================

\dots


%% ===========================
\section{Zweiter Abschnitt}
\label{ch:Content1:sec:Section2}
%% ===========================

\dots


%% content.tex
%%

%% ==============
\chapter{Ein Kapitel}
\label{ch:Content2}
%% ==============

\dots


%% ===========================
\section{Erster Abschnitt}
\label{ch:Content2:sec:Section1}
%% ===========================

\dots


%% ===========================
\section{Zweiter Abschnitt}
\label{ch:Content2:sec:Section2}
%% ===========================

\dots




%% --------------------
%% |   Bibliography   |
%% --------------------
\cleardoublepage
\phantomsection
\addcontentsline{toc}{chapter}{\bibname}

\iflanguage{english}
{\bibliographystyle{IEEEtranSA}}	% english style
{\bibliographystyle{babalpha-fl}}	% german style
												  
% Use IEEEtran for numeric references
%\bibliographystyle{IEEEtranSA})

\bibliography{ausarbeitung}
\Erklaerung
\end{document}
