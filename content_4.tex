\section{Introduction | Einleitung}
Mandatory. Questions like: What is the topic of this work, what's the broader context (topic of the seminar), why is it relevant?


\section{Methods/Algorithms/Derivations | Methoden/Algorithmen/Herleitungen}
If applicable.
\subsection{Method 1}
If applicable.


\section{Applications/Examples/Own experiments | Anwendung/Beispiele/Eigene Experimente}
If applicable.
\subsection{Application 1}
If applicable.


\section{Summary and conclusion | Zusammenfassung und Fazit}
Mandatory. Short summary of the most important aspects of the report.
If possible: What are open challenges?

\newpage
\section{\LaTeX Examples}
As a help to get started with this template. To be deleted for submission.
\subsection{Citation examples}
\citet{campbell:2017} define the stages of information processing in a nervous system as: "sensory input, integration, and motor output". \\
The stages of information processing in a nervous system are defined as: "sensory input, integration, and motor output" \citep{campbell:2017}. 

\subsection{Table example}
\input{tables/random_numbers}

\subsection{Figure examples}
This is a png file, it gets blurry when you zoom in:
\begin{figure}[htbp]
    \centering
    \includegraphics[width=.7\textwidth]{figures/leaky_integration.png}
    \caption{Symbolic representation of a leaky integrating neuron.}
    \label{fig:leaky_integration}
\end{figure}

This is an eps file, it is always sharp:\\
Notice how the formatting option "[htbp]" allows for the figure to be moved around to page \pageref{fig:activation_function}. Hence, it is best to rather write: The eps file in figure \ref{fig:activation_function} always stays sharp.
\begin{figure}[htbp]
    \centering
    \includegraphics[width=.7\textwidth]{figures/activation_functions}
    \caption{Shapes of a parametrized tanh activation function.}
    \label{fig:activation_function}
\end{figure}

\subsection{Math example}
The state update of the leaky integrating neuron in figure \ref{fig:leaky_integration} can be formulated as:
\begin{align}
    x_i(t+1) &= \lambda_i \cdot \left(W_{i,j} \cdot U_j(t)\right) + (1-\lambda_i) \cdot \theta_i(t)
    \label{eq:leaky_integration}
\end{align}

\subsection{Code block example}
From equation \ref{eq:leaky_integration} the neuron model is implemented using numpy \citep{harris:2020}:

\lstinputlisting[label=py:leaky, language=Python, caption=Python implementation of a single leaky-integrating neuron.]{code/leaky.py}

\subsection{Footnote example}
The implementation is available on github\footnote{https://github.com/schniewmatz/recurrence}.