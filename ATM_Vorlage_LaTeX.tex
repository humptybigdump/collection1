\documentclass[oneside,a4paper,12pt,table]{scrreprt} % Gibt an: Papierformat, Schriftgröße

\usepackage[table]{xcolor}
\usepackage{FormatierungenVorlage}

% Schriftart
\usepackage{lmodern} 
%\usepackage[center]{caption}
\usepackage[nooneline]{caption}

% Zitierstil
\usepackage[natbibapa,numberedbib,sectionbib]{apacite}
\bibliographystyle{apacite}


\begin{document}
	
% Sprache einstellen.
\setlanguageGerman 
\setlength{\parindent}{1cm}


% Hier kommt der ganze Vorspann wie Abkürzungs-, Abbildungs- oder Tabellenverzeichnisse
\include{sections/preamble}

% Füge hier die Kapitel per Verweis ein. Der Befehl \include fügt die angegebene tex-Datei an der jeweiligen Stelle ein. Die eingefügte Datei wird als normaler Teil des Quelltextes mitverarbeitet und ist daher LaTeX-Code (ohne Vorspann und \begin{document}...\end{document}).

\include{sections/Einleitung}
\include{sections/Forschungsfragen}
\chapter{Diskussion}
\include{sections/Fazit}

% Literaturverzeichnis
\newpage
\addcontentsline{toc}{chapter}{Literaturverzeichnis}
% Datei mit Literaturangaben einbinden (am einfachsten Quellen aus Citavi als BibTeX-Datei exportieren)
%\bibliography{ATM_Quellen} 

\end{document}