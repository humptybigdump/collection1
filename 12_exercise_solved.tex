\input{../exercise_preamble.tex}

\usepackage{multirow}

\begin{document}

\author{}
\date{}
\title{Optimization Methods for \\Machine Learning and Engineering\\\vspace{0.5cm}\textit{Exercise 9}}
\maketitle

%\printallsolutions*[]

\setcounter{section}{9}
\setcounter{exercise}{0}


\begin{exercise}[subtitle={Paper}]
Consider the problem:
      \begin{align*}
          \min_{\vec x\in \mathbb R^ 2} \quad & f(\vec x) = x_1^2 + 3 x_2^2\\
          \text{subject to} \quad & x_1+2x_2-1 = 0
      \end{align*}
            \begin{enumerate}[label=\emph{\alph*)}]
\item Write down the Lagrangian and the Lagrangian dual function,
\item Write down the KKT conditions for the problem.
\item Write down the Slater condition for the problem.
\item Find the solution by maximizing the Lagrangian dual.
\end{enumerate}
\end{exercise}

\begin{solution}[print=false]
        The Lagrangian $\lagrangian$ of the optimization problem is
        $\lagrangian(\vec x, \lambda) = x_1^2+ 3x_2^2{} + \lambda(x_1+2x_2-1)$
  
        \textbf{Constrained:} Optimum at $\grad_{\vec x}{\lagrangian} = \vec 0 $ and $ \grad_\lambda{\lagrangian} = 0$


    \begin{align}
        \grad_{\vec x} \lagrangian(\vec x, \lambda) = \begin{pmatrix}
          \begin{aligned}
            2x_1 &+ \lambda\\
           6x_2 &+ 2\lambda
          \end{aligned}
        \end{pmatrix} = \begin{pmatrix} 0\\0 \end{pmatrix}
& \Rightarrow \quad 2x_1= 3x_2 = -\lambda\\[10pt]
        \grad_\lambda \lagrangian(\vec x, \lambda) = x_1+2x_2-1=0 &\Rightarrow \quad x_1 = \tfrac{3}{7},\ x_2 = \tfrac{2}{7},\ \lambda = -\tfrac{6}{7}
      \end{align}
      
\end{solution}

\begin{exercise}[subtitle={Bonus: Paper}]
Is the Lagrangian dual function $q(\vec \lambda)$ convex or concave? Explain your answer.
\end{exercise}

\begin{solution}[print=false]
Since the Lagrange dual function is the pointwise infimum of a family of affine functions of $(\lambda, \mu)$, it is
concave, even when the problem is not convex.
\end{solution}


\begin{exercise}[subtitle={Paper}]
%Derive surrogate duality gap for feasible point: $m/t$. Note that this duality gap only applies if every point $x_k$ is feasible.\\
Recall the central path generated by the interior point method discussed in Exercise 5.1: \\
Let $f(\vec{x}): \mathbb{R}^n \rightarrow \mathbb{R}$ and $g_j(\vec{x}): \mathbb{R}^n \rightarrow \mathbb{R}\; (j = 1,\dots, m)$ be convex functions which are twice differentiable.\\
 Let $\{\vec{x}^{(k)}\}$ be a sequence generated by an interior point method with barrier function $\phi(\vec{x}) = -\sum_{j = 1}^m\log\left(-g_j(\vec{x})\right)$ by solving the optimization problem
\begin{equation}
\vec{x}^{(k)} = \myargmin_{\vec{y}\in \mathbb{R}^n}\left[f(\vec{y}) + \frac{1}{t^{(k)}}\phi(\vec{y})\right]
\label{eq:central_path}
\end{equation}
with $t^{(k)} \in \mathbb{R}, t^{(k)} > 0$. $t^{(k)}$ increases with each iteration $k \in \mathbb{N}$ such that $\lim_{k\rightarrow\infty} t^{(k)} = \infty$.\\
Consider the explicit problem   \begin{align*}
    \min_{\vec x} \quad & f(\vec x) = \vec{c}^{\top} \vec{x}\\
    \text{subject to} \quad & \vec g(\vec x) =  \mat A \vec{x} - \vec{b} \leq \vec{0}
  \end{align*}

\begin{enumerate}[label=\emph{\alph*)}]
\item Write down the Lagrangian and the Lagrangian dual function $q(\vec \mu)$ with $\vec \mu = (\mu_1,\dots, \mu_m)^\top $. \label{dual_func}
\item Write down the Slater condition for the given example.
\item Write down the optimality conditon for $\vec{x}^{(k)}$ being a solution of the problem stated in equation (\ref{eq:central_path}). Let's define \begin{equation}
\mu^{(k)}_i = -\frac{1}{t^{(k)} g_i(\vec x^{(k)})}~.
\label{eq:mu_feasible}
\end{equation}
\item Plug $\vec \mu^{(k)} = (\mu_1^{(k)},\dots, \mu_m^{(k)})^\top $ generated by the interior point method into your Lagrangian from \ref{dual_func}. Note that equation (\ref{eq:mu_feasible}) only holds at the point $\vec x = \vec{x}^{(k)}$. \label{plug}
\item Evaluate $\eta^{(k)} = \sum_{i = 1}^m g_i(\vec x^{(k)})\mu_i^{(k)}$ and explain its relation to the solution in \ref{plug}. What information does $\eta^{(k)}$ hold? \label{gap}
%\item Compare the solutions from \ref{plug} and \ref{gap} and explain their relation.
\end{enumerate}
\end{exercise}

\begin{solution}[print=false]
Recall the optimality condition for a point $x^{(k)}$
\begin{align}
\grad f(\vec x^{(k)}) + \grad \left(\frac{1}{t^{(k)}}\phi(\vec x^{(k)})\right) = 0\\
\mu^{(k)}_i = -\frac{1}{t g_i(\vec x^{(k)})}\\
\grad f(\vec x^{(k)}) + \sum \mu^{(k)}_i \grad f(\vec x^{(k)}) = 0\\
\end{align}
We see that $\vec x_k$ minimized the Lagrangian
\begin{align*}
L(x,\mu) = f(\vec x) + \sum_{i = 1}^{m} \mu_i g_i(\vec x^{(k)})
\end{align*}
for $\mu_i = \mu_i^{(k)}$. This implies that $\vec x^{(k)}$ generates a dual feasible variable.
Evaluate the dual function $q(\mu^{(k)})$
\begin{align*}
q(\vec \mu^{(k)}) &= f(\vec x^{(k)}) + \sum_{i = 1}^{m} \mu^{(k)}_i g_i(\vec x^{(k)})\\
& = f(\vec x^{(k)}) - \frac{m}{t}
\end{align*}
\begin{itemize}
\item $\eta^{(k)} = \sum_{i = 1}^m g_i(\vec x^{(k)})\mu_i^{(k)}$ is the \emph{surrogate duality gap}. While the term \emph{duality gap} is defined as the gap between the optima of the primal and the dual problem, the \emph{surrogate duality gap} quantifies the gap for suboptimal points generated in numerical solvers.
\end{itemize}
\end{solution}



\begin{exercise}[subtitle={Notebook}]
Follow the instructions in the notebook and implement the primal-dual interior point method.
\end{exercise}




\end{document}