\chapter{Übung 0}
\setcounter{page}{1}

\section{Fragestellung}
In dieser Übung wird die Rückstands- und die Produktzusammensetzung einer absatzweisen Destillation eines Dreikomponentengemisches berechnet. In Abbildung \ref{fig:Trennstufe} ist die Trennstufe schematisch dargestellt.

\begin{figure}[h]
    \centering
    \includegraphics[scale=0.7]{img/Trennstufe_cropped.pdf}
         \caption{Trennstufe}
    \label{fig:Trennstufe}
\end{figure}

\noindent Der Verlauf der Destillation wird über das Verhältnis der Anfangsstoffmenge zur aktuellen Stoffmenge dargestellt:

\begin{equation}
 \frac{N_{\text{L}}}{N_{\text{L,0}}}   
\end{equation}

\noindent Die Ausgangszusammensetzung sowie die Trennfaktoren sind in Tabelle \ref{tab:Vorgaben} angegeben.

\begin{table}[h]
    \centering
    \caption{Anfangszusammensetzung und Trennfaktoren} 
    \begin{tabular}{l| c}
      $ \tilde x_{1,0}$ & 0,2 \\ \hline
      $ \tilde x_{2,0}$ & 0,55 \\ \hline
      $ \tilde x_{3,0}$ & 0,25 \\ \hline
      $ \omega_{2,1}$ & 0,4 \\ \hline
      $ \omega_{3,1}$ & 0,2 
        \end{tabular}
    \label{tab:Vorgaben}
\end{table}


\section{Bilanzen}
\label{ch:Bilanz}
In diesem Kapitel werden die Berechnungsvorschriften für die verschiedenen Zusammensetzungen hergeleitet.

\subsection{Rückstandszusammensetzung}

Zur Berechnung der Rückstandszusammensetzung wird eine Gesamtbilanz und pro Komponente eine Komponentenbilanz erstellt. Als Bilanzraum wird dabei die Trennstufe gewählt. Durch Einführung des Trennfaktors $\omega_{\text{ij}} = \frac{\tilde y_{\test{j}}}{\tilde x_{\test{j}}} \cdot \Big( \frac{\tilde y_{\test{i}}}{\tilde x_{\test{i}}} \Big) ^{-1}$ und Lösung des Gleichungssystem werden die nachfolgenden Gleichung bestimmt.

\noindent Die Berechnung der Rückstandszusammensetzung $\tilde x_1$ erfolgt mit Gleichung \ref{eq:Kompo_1}.

\begin{equation}
    1 = \tilde x_{\text{1}} + \tilde x_{\text{2}} \cdot \Big( \frac{\tilde x_{\text{1}}}{\tilde x_{\text{1,0}}} \Big) ^{\omega_{\text{21}}} \cdot \Big( \frac{N^{\text{L}}}{N_{0}^{\text{L}}} \Big) ^{\omega_{\text{21 -1}}} + \tilde x_{\text{3}} \cdot \Big( \frac{\tilde x_{\text{1}}}{\tilde x_{\text{1,0}}} \Big) ^{\omega_{\text{31}}} \cdot \Big( \frac{N^{\text{L}}}{N_{0}^{\text{L}}} \Big) ^{\omega_{\text{31 -1}}}
    \label{eq:Kompo_1}
\end{equation}

\noindent Gleichung \ref{eq:Kompo_1} kann analytisch nicht gelöst werden. Die nummerische Lösung ist in Abschnitt \ref{sec:Pyth} dargestellt.

\noindent Die Rückstandszusammensetzung von Komponente 2 und 3 kann explizit über Gleichung~\ref{eq_Kompo_23} berechnet werden.

\begin{equation}
    \tilde x_{\text{i}}^{\text{L}} = \tilde x_{\text{i,0}} \cdot \Big( \frac{\tilde x_{\text{1}}}{\tilde x_{\text{1,0}}} \Big) ^{\omega_{\text{i1}}} \cdot \Big( \frac{N^{\text{L}}}{N_{0}^{\text{L}}} \Big) ^{\omega_{\text{i1 -1}}}
    \label{eq_Kompo_23}
\end{equation}

\subsection{Fraktioniertes Produkt}
Die Berechnung der Zusammensetzung des fraktionierten Produktes erfolgt über die Trennfaktoren und die Zusammensetzung des Rückstandes. Aus der Definition des Trennfaktors folgt Gleichung \ref{eq:Trenn}.

\begin{equation}
    \tilde y_{\text{j}} = \tilde y_{\text{p}} \cdot \omega_{\text{jp}} \cdot \frac{\tilde x_{\text{j}}}{\tilde x_{\text{p}}}
    \label{eq:Trenn}
\end{equation}

\noindent Mit der Summationsgleichung (Summe aller Molanteile = 1) und Gleichung \ref{eq:Trenn}, kann der Molanteil der Komponente 1 im Produktstrom nach Gleichung \ref{eq:Produkt} berechnet werden.

\begin{equation}
    \tilde y_1 = \Big( 1 + \omega_{21} \cdot \frac{\tilde x_{2}}{\tilde x_1}  + \omega_{31} \cdot \frac{\tilde x_{3}}{\tilde x_1} \Big) ^{-1}
    \label{eq:Produkt}
\end{equation}

\noindent Die Molanteile von Komponente 2 und 3 werden über Gleichung \ref{eq:Trenn} berechnet.


\subsection{Kumuliertes Produkt}

Für die Berechnung der kumulierten Produktzusammensetzung wird eine Gesamt- und eine Komponentenbilanz aufgestellt. Dabei wird als Bilanzraum die Blase und das kondensierte Destillat gewählt. Durch einsetzten der Gesamtbilanz in die Komponentenbilanz ergibt sich Gleichung \ref{eq:kum_Prod}. Mit dieser wir die kumulierte Produktzusammensetzung berechnet werden kann.

\begin{equation}
    \overline{\tilde x_{\text{i,D}}} = \frac{1}{1 - \frac{N^{\text{L}}}{N^{\text{L}}_0}} \cdot \tilde x_{\text{i,0}} - \frac{\frac{N^{\text{L}}}{N^{\text{L}}_0}}{1 - \frac{N^{\text{L}}}{N^{\text{L}}_0}} \cdot \tilde x_{\text{i}}
    \label{eq:kum_Prod}
\end{equation}


\section{Numerische Umsetzung mit Python}
\label{sec:Pyth}

Zur Berechnung der Zusammensetzungen und Erstellung der Diagramme, wurde ein Programm in Python erstellt. Im folgenden Kapitel wird kurz auf die Struktur des Programmes, sowie auf einzelnen Funktionen eingegangen.
In Abbildung \ref{fig:Klassen} ist eine Übersicht der einzelnen Klassen dargestellt. Dabei sind die Klassen \textit{ResidueComposition,  ProductComposition} \textit{ProductCompositionCumulated} für die Berechnung der Molanteile nach den Gleichungen aus Kapitel \ref{ch:Bilanz} zuständig. Die Klasse \textit{MatrixBuilder} erstellt jeweils die Matrizen in welchen die jeweiligen Molanteile und der dazugehörige Wert von $\frac{N^{\text{L}}}{N_{0}^{\text{L}}}$ abgespeichert werden. Die Klassen \textit{Diagram} und \textit{TernaryDiagram} sind für die Erzeugung der Diagramme und der Dreiecksdiagramm zuständig.

\begin{figure}[h]
    \centering
    \includegraphics[scale=0.8]{img/Uebersicht_Klassen.PNG}
         \caption{Klassenübersicht}
    \label{fig:Klassen}
\end{figure}

\clearpage

\noindent Als Eingabeparameter benötigt das Programm die Ausgangszusammensetzung, die Trennfaktoren sowie die Schrittweite für die numerische Berechnung.

\noindent Die numerische Lösung von Gleichung \ref{eq:Kompo_1} erfolgt mit dem Newton Verfahren. Dafür wird die Funktion \textit{optimize.newton} aus der Bibliothek \textit{SciPy} genutzt. Die Implementierung ist in Abbildung \ref{fig:Newton} dargestellt. Als Startwert für das Newton Verfahren wird der Wert des vorherigen Schrittes gewählt. Für die Toleranz wird ein fester Wert von 1E-7 gewählt.

\begin{figure}[h]
    \centering
    \includegraphics[scale=0.7]{img/newton.PNG}
         \caption{Implementierung optimize.newton}
    \label{fig:Newton}
\end{figure}



\clearpage

\section{Ergebnisse}
In diesem Kapitel sind alle Ergebnisdiagramme dargestellt. Für die Berechnung wurde eine Schrittweite von 0.01 gewählt.  

\subsection{Rückstandszusammensetzung}

\begin{figure}[h]
    \centering
    \includesvg[scale=0.63]{img/Rueck_Dia}
        \caption{Diagramm Rückstandszusammensetzung}
    \label{fig:Trennstufe}
\end{figure}

\begin{figure}[h]
    \centering
    \includesvg[scale=0.63]{img/Rueck_Dreieck}
        \caption{Dreiecksdiagramm Rückstandszusammensetzung}
    \label{fig:Trennstufe}
\end{figure}

\subsection{Produktzusammensetzung}

\begin{figure}[H]
    \centering
    \includesvg[scale=0.63]{img/Prod_Dia}
        \caption{Diagramm Produktzusammensetzung}
    \label{fig:Trennstufe}
\end{figure}

\begin{figure}[H]
    \centering
    \includesvg[scale=0.63]{img/Prod_Dreieck}
        \caption{Dreiecksdiagramm Produktzusammensetzung}
    \label{fig:Trennstufe}
\end{figure}


\subsection{Kumulierte Produktzusammensetzung}

\begin{figure}[H]
    \centering
    \includesvg[scale=0.63]{img/kum_Prod_Dia}
        \caption{Diagramm kumulierte Produktzusammensetzung}
    \label{fig:Trennstufe}
\end{figure}

\begin{figure}[H]
    \centering
    \includesvg[scale=0.63]{img/kum_Prod_Dreieck}
        \caption{Dreiecksdiagramm kumulierte Produktzusammensetzung}
    \label{fig:Trennstufe}
\end{figure}
