\input{../exercise_preamble.tex}

\begin{document}

\author{}
\date{}
\title{Optimization Methods for \\Machine Learning and Engineering\\\vspace{0.5cm}\textit{Exercise 12}}
\maketitle

%\printallsolutions*[]

\setcounter{section}{12}
\setcounter{exercise}{0}


\begin{exercise}[subtitle={Notebook}]
Consider the example of the optimal portfolio selection at the stock market. Follow the instructions in the notebook, compute the relevant statistical quantities and optimize your portfolio.
\end{exercise}

%\begin{exercise}[subtitle={Paper + Notebook}]
%Consider the example of the inverted pendulum (Lecture 5, slide 33) where the pendulum state is described by the vector
%\begin{equation}
%\vec x_t = [\theta_t,p_t]
%\end{equation}
%The acceleration of the angular acceleration $\ddot{\theta}$ and the cart acceleration $\ddot{p}$ pendulum are given by:
%\begin{align}
%  \ddot \theta_t & = \frac{(c+m)g\sin(\theta_t)-ml\sin(\theta_t)\cos(\theta_t)\dot\theta_t^2-\cos(\theta_t) u_t}{l(c+m\sin^2(\theta_t))}\label{eq:theta} \\
%  \ddot p_t &=\frac{-mg\sin(\theta_t)\cos(\theta_t)+ml\sin(\theta_t)\dot\theta_t^2+u_t}{c+m\sin^2(\theta_t)}\label{eq:p}
%\end{align}
%Given the state $\vec{x}_t$, the state at time $t+1$ can be approximated using the Verlet method by:
%\begin{align}
%\vec x_{t+1} = 2 \vec x_t - \vec{x_{t-1}} + \ddot{\vec x}_t \frac{\Delta t^2}{2} \label{eq:verlet}
%\end{align}
%
%\begin{enumerate}[label=\emph{\alph*)}]
%\item \textit{Paper:} Insert equations (\ref{eq:theta}) and (\ref{eq:p}) into (\ref{eq:verlet}) and bring it into the shape $\vec x_{t+1} = \mat{A}_t \vec x_t + \vec b_t u_t + \vec c_t$
%\item \textit{Notebook:} Implement the calculation of $\mat{A}_t$,$\vec b_t$ and $\vec c_t$. Simulate the pendulum.
%\item \textit{Paper:} Consider the time-dependent pendulum states $\vec{x}_t$ extended by the control $u_t$ and stacked into one vector $\tilde{\vec x}_T$: 
%\begin{align*}
%\tilde{\vec x}_T = \begin{pmatrix}
%\theta_1\\
%p_1\\
%u_1\\
%\theta_2\\
%p_2\\
%u_2\\
%\vdots\\
%u_T
%\end{pmatrix}
%\end{align*}
%We want to solve the following optimization problem for the prediction horizon $T$:
%\begin{align*}
%&\min_{\tilde{\vec{x}_T}} \tilde{\vec x}_T^\top \mat Q \tilde{\vec{x}_T}\\
%\text{subject to} \quad & \tilde{\mat A}\tilde{\vec x}_T = \tilde{\vec b}
%\end{align*}
%Write down the structure of the matrices $\tilde{\mat A}$ and $\mat Q$ and the vector $\tilde{\vec b}$ using $A_t$, $b_t$ and $c_t$. Which sizes do they have and what entries do they contain?
%\item \textit{Notebook:} Implement the model predictive control algorithm.
%\end{enumerate}
%\end{exercise}


\begin{exercise}[subtitle={Paper + Notebook}]
Consider the example of the inverted pendulum where the pendulum state is described by the vector
\begin{equation}
\vec x_t = [\theta_t,p_t].%,\dot{\theta}_t,\dot{p}_t].
\end{equation}
The acceleration of the angular acceleration $\ddot{\theta}$ and the cart acceleration $\ddot{p}$ pendulum are given by:
\begin{align}
  \ddot \theta_t & = \frac{(c+m)g\sin(\theta_t)-ml\sin(\theta_t)\cos(\theta_t)\dot\theta_t^2-\cos(\theta_t) u_t}{l(c+m\sin^2(\theta_t))}\label{eq:theta} \\
  \ddot p_t &=\frac{-mg\sin(\theta_t)\cos(\theta_t)+ml\sin(\theta_t)\dot\theta_t^2+u_t}{c+m\sin^2(\theta_t)}\label{eq:p}
\end{align}
Given the state $\vec{x}_t$, the state at time $t+1$ can be approximated by:
\begin{equation}
\label{eq:verlet}
\begin{aligned}
\vec x_{t+1} &= \vec x_t + \dot{\vec{x}}_{t} \Delta t + \ddot{\vec x}_t \frac{\Delta t^2}{2} \\
\dot {\vec x}_{t+1} &= \dot{\vec x}_t + \ddot{\vec{x}}_{t} \Delta t
\end{aligned}
\end{equation}
For this consider the extended state $\vec z_t = \begin{pmatrix} \vec x_t\\ \dot{\vec x}_t\end{pmatrix}$.

\begin{enumerate}[label=\emph{\alph*)}]
\item \textit{Paper:} Insert equations (\ref{eq:theta}) and (\ref{eq:p}) into (\ref{eq:verlet}) and bring it into the shape $\vec z_{t+1} = \mat{A}_t \vec z_t + \vec b_t u_t + \vec c_t$
\item \textit{Notebook:} Implement the calculation of $\mat{A}_t$,$\vec b_t$ and $\vec c_t$. Simulate the pendulum.
\item \textit{Paper:} To convert the problem to an optimization problem, the time-dependent pendulum states $\vec{x}_t$ are extended by the control $u_t$ and stacked into one vector $\tilde{\vec x}_T$: 
\begin{align*}
\tilde{\vec x}_T = \begin{pmatrix}
\theta_1\\
p_1\\
\dot{\theta_1}\\
\dot{p_1}\\
u_1\\
\theta_2\\
p_2\\
\vdots\\
\dot{\theta_T}\\
\dot{p_T}\\
u_T
\end{pmatrix}
\end{align*}
We want to solve the following optimization problem for the prediction horizon $T$:
\begin{align*}
&\min_{\tilde{\vec{x}_T}} \tilde{\vec x}_T^\top \mat Q \tilde{\vec{x}_T}\\
\text{subject to} \quad & \tilde{\mat A}\tilde{\vec x}_T = \tilde{\vec b}
\end{align*}
Write down the structure of the matrix $\tilde{\mat A}$ and the vector $\tilde{\vec b}$ using $\mat A_t$, $\vec b_t$ and $\vec c_t$.
\item \textit{Notebook:} Implement the model predictive control algorithm.
\end{enumerate}
\end{exercise}


\begin{solution}[print=false]

\begin{align*}
\vec x_{t+1} = A_t \vec x_t + \vec b_t  u_t + \vec c_t\\
\mat A_t = \begin{pmatrix}
1 & 0 & \Delta t & 0\\
0 & 1 & 0  & \Delta t\\
0 & 0 & 1  & 0\\
0 & 0 & 0  & 1
\end{pmatrix}\\
\vec b_t = \begin{pmatrix}
\Delta t \frac{-\cos(\theta_t)}{l(c+m \sin^2(\theta_t))} \frac{\Delta t}{2}\\
\frac{\Delta t} {(c+m \sin^2(\theta_t))} \frac{\Delta t}{2}\\
\Delta t \frac{-\cos(\theta_t)}{l(c+m \sin^2(\theta_t))}\\
\frac{\Delta t} {(c+m \sin^2(\theta_t))}
\end{pmatrix}\\
\vec{c}_t = 
\begin{pmatrix}
\Delta t \frac{(c+m) g \sin(\theta_t)-m l \sin(\theta_t) \cos(\theta_t) \dot{\theta}_t^2}{l (c+m*sin^2(\theta))} \frac{\Delta t}{2}\\
\Delta t \frac{-m g \sin(\theta_t) \cos(\theta_t)+m l \sin(\theta_t) \dot{\theta}_t^2}{c+m \sin^2(\theta_t)} \frac{\Delta t}{2}\\
\Delta t \frac{(c+m) g \sin(\theta_t)-m l \sin(\theta_t) \cos(\theta_t) \dot{\theta}_t^2}{l (c+m*sin^2(\theta))}\\
\Delta t \frac{-m g \sin(\theta_t) \cos(\theta_t)+m l \sin(\theta_t) \dot{\theta}_t^2}{c+m \sin^2(\theta_t)}
\end{pmatrix}
\end{align*}
Different formulation when considering the $\dot{\theta}_t^2$ in $\vec{c}_t$ as being linear in $\dot{\theta}_t$:
\begin{align*}
\mat A_t = \begin{pmatrix}
1 & 0 & \Delta t + \frac{\Delta t}{2} \frac{- \Delta t (m l \sin(\theta_t) \cos(\theta_t) \dot{\theta}_t)}{l (c+m \sin^2(\theta_t))} & 0\\
0 & 1 & \Delta t \frac{m l \sin(\theta_t) \dot{\theta}_t}{c+m \sin^2(\theta_t)} \frac{\Delta t}{2} & \Delta t\\
0 & 0 & 1 + \frac{- \Delta t (m l \sin(\theta_t) \cos(\theta_z) \dot{\theta_t})}{l (c+m \sin^2(\theta_t))}  & 0\\
0 & 0 & \frac{- \Delta t (m l \sin(\theta_t) \cos(\theta_t) \dot{\theta_t})}{l (c+m \sin^2(\theta_t))}  & 1
\end{pmatrix}\\
\vec b_t = \begin{pmatrix}
\Delta t \frac{-\cos(\theta_t)}{l(c+m \sin^2(\theta_t))} \frac{\Delta t}{2}\\
\frac{\Delta t} {(c+m \sin^2(\theta_t))} \frac{\Delta t}{2}\\
\Delta t \frac{-\cos(\theta_t)}{l(c+m \sin^2(\theta_t))}\\
\frac{\Delta t} {(c+m \sin^2(\theta_t))}
\end{pmatrix}\\
\vec{c}_t = 
\begin{pmatrix}
\Delta t \frac{(c+m) g \sin(\theta_t)}{l (c+m \sin^2(\theta_t))} \frac{\Delta t}{2}\\
\Delta t \frac{-m g \sin(\theta_t) \cos(\theta_t)}{c+m \sin^2(\theta_t)} \frac{\Delta t}{2}\\
\Delta t \frac{(c+m) g \sin(\theta_t)}{l (c+m \sin^2(\theta_t))}\\
\Delta t \frac{-m g \sin(\theta_t) \cos(\theta_t)}{c+m \sin^2(\theta_t)}
\end{pmatrix}
\end{align*}
\end{solution}

\end{document}