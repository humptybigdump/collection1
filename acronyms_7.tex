\pagestyle{plain}
\chapter*{Abkürzungsverzeichnis}
\addcontentsline{toc}{chapter}{Abkürzungsverzeichnis}

\noindent Abkürzungen sind sparsam zu verwenden und müssen vor der ersten Nutzung im Text erläutert werden. Dazu wird nach dem vollen Wortlaut in runden Klammern die Abkürzung eingeführt. 
Danach ist diese konsistent zu verwenden, da ein Wechsel von Langform und Abkürzung den Leser unnötig verwirrt. 
Abkürzungen, die auch im Duden aufgeführt sind, dürfen ohne weitere Erklärung verwendet werden. 
Exemplarisch seien hier z.B., usw., etc., oder IQ genannt. Auch metrische und nonmetrische Maßeinheiten wie cm, kg, oder min sind gebräuchlich und somit ohne Erklärung nutzbar. 
Alle anderen Abkürzungen müssen im Abkürzungsverzeichnis mit vollem Wortlaut aufgeführt werden, z.B.:

%TODO Abkürzungen und Langform einfügen
\begin{acronym}[DGPs]
	\acro{ATM} {Arbeitstechniken im Maschinenbau}
	\acro{DGPs}{Deutsche Gesellschaft für Psychologie}
	\acro{APA}{American Psychological Association}
\end{acronym}