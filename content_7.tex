\section{Introduction | Einleitung}
Mandatory. Questions like: What is the topic of this work, what's the broader context (topic of the proseminar), why is it relevant? When/by whom was the algorithm invented? What were the goals?


\section{Algorithm/principal ideas behind the method |Algorithmus/Hauptideen hinter der Methode}
If applicable.
\subsection{Algorithm}
If applicable.

\subsection{Main idea}
If applicable (the gist). 
\subsection{Example implementation}
Code examples are useful to see how algorithms are actually implemented. May use available software packages such as scikit-learn, keras, Tensorflow etc. Python is a very common language in machine learning, but e.g. Julia is also becoming popular.
\subsection{Advantages and disadvantages}
Compared to e.g. other regression or classification algorithms.
What is it doing and why is it doing it so (relatively) well? However, also, what are potential weaknesses/blind spots?

\section{Applications in Climate and Environmental Sciences | Anwendungsbeispiele aus dem Themenbereich}
\subsection{Application 1}
Example where and why is this useful. Critical reflection - why this algorithm for this particular use case?

\subsection{Application 2}
Example where and why is this useful. Critical reflection - why this algorithm for this particular use case?


\section{Summary and conclusion | Zusammenfassung und Fazit}
Mandatory. Short summary of the most important aspects of the report.
If possible: What are open challenges?


\newpage
\section{\LaTeX Examples}
As a help to get started with this template. To be deleted for submission.\\
To compile the document, go to your command line and run biblatex main.tex, biber main, biblatex main.tex. This will generate a document called main.pdf.

\subsection{Citation examples}
Urry et al. \cite{campbell:2017} define the stages of information processing in a nervous system as: "sensory input, integration, and motor output". \\
The stages of information processing in a nervous system are defined as: "sensory input, integration, and motor output" \cite{campbell:2017}. 


\subsection{Table example}
Table~\ref{tab:random} shows a table.
\input{tables/random_numbers}


\subsection{Figure examples}
Figure~\ref{fig:leaky_integration} shows a png file.
\begin{figure}[htbp]
    \centering
    \includegraphics[width=.7\textwidth]{figures/leaky_integration.png}
    \caption{Symbolic representation of a leaky integrating neuron.}
    \label{fig:leaky_integration}
\end{figure}


\subsection{Math example}
The state update of the leaky integrating neuron in figure \ref{fig:leaky_integration} can be formulated as:
\begin{align}
    x_i(t+1) &= \lambda_i \cdot \left(W_{i,j} \cdot U_j(t)\right) + (1-\lambda_i) \cdot \theta_i(t)
    \label{eq:leaky_integration}
\end{align}


\subsection{Code block example}
From Equation~\ref{eq:leaky_integration} the neuron model is implemented using numpy \cite{harris:2020}:
\lstinputlisting[label=py:leaky, language=Python, caption=Python implementation of a single leaky-integrating neuron.]{code/leaky.py}


\subsection{Footnote example}
The implementation is available on github\footnote{https://github.com/aimat-lab}.


