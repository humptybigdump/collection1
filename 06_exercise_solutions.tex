\input{../exercise_preamble.tex}

\begin{document}

\author{}
\date{}
\title{Optimization Methods for \\Machine Learning and Engineering\\\vspace{0.5cm}\textit{Exercise 6}}
\maketitle

%\printallsolutions*[]

\setcounter{section}{6}
\setcounter{exercise}{0}


%\begin{exercise}[subtitle={Paper}]
%Show that the intersection of an infinite number of convex sets is convex.
%\end{exercise}



\begin{exercise}[subtitle={Paper}]
%The optimization problem
%\begin{align*}
%\text{minimize}\quad &f(\vec{x})\\
%\text{subject to}\quad & g_j(\vec{x})\leq 0,\quad j = 1,\dots, m\\
% & h_i(\vec{x})= 0,\quad i = 1,\dots, k
%\end{align*}
%is convex iff $f(\vec{x})$ is convex, $g_j(\vec{x})$ is convex $\forall j \in {1,\dots,m}$ and $h_i(\vec{x})$ is affine $\forall i \in {1,\dots,k}$.\\
Consider the following optimization problem with $\vec{x}\in \mathbb{R}^2$ and $\vec{x}^\top = (x_1,x_2)$:
\begin{align*}
\text{minimize}\quad &f(\vec{x}) = x_1^2 + x_2^2\\
\text{subject to}\quad & g(\vec{x}) = x_1/(1+x_2^2)\leq 0,\\
& h(\vec{x}) = (x_1 + x_2)^2 = 0.
\end{align*}
Is this a convex optimization problem? If not, can you reformulate it into an equivalent problem (same optimal point) which fulfills all criteria of a convex optimization problem?
\end{exercise}


\begin{solution}[print=true]
\begin{align*}
\text{minimize}\quad &f(\vec{x}) = x_1^2 + x_2^2\\
\text{subject to}\quad & g(\vec{x}) = x_1\leq 0\\
& h(\vec{x}) = x_1 + x_2 = 0
\end{align*}
\end{solution}


\begin{exercise}[subtitle={Paper + Notebook}]
  A company produces different paint for interior and exterior use. Paint
  production uses the raw materials A and B. The production of one ton of
  interior paint uses two tons of A and one ton of B. The production of one ton
  of exterior paint uses up one ton of A and two tons of B. The availability of
  the raw material is 6 tons of A and 7 tons of B per day.

  The market price is 2k\texteuro{} per ton of interior paint and 3k\texteuro{}
  per ton of exterior paint. A market study has predicted a daily sales
  potential of two tons for interior paint and three tons for exterior paint.\\\\
  What are the daily production rates that maximize revenue?

\begin{enumerate}[label=\emph{\alph*)}]
\item \textit{Paper:} State the optimization problem as a linear program of the form:
  \begin{align*}
    \min_{\vec x} \quad & \vec{c}^{\top} \vec{x}\\
    \text{subject to} \quad & \mat A \vec{x} - \vec{b} \leq \vec{0}
  \end{align*}
\item \textit{Paper:} Sketch the feasible region.
\item \textit{Notebook:} Solve the problem using the interior point method.
\item \textit{Notebook:} Plot the precision of the intermediate solutions after each outer iteration versus the cumulative number of newton iterations needed in the inner iteration. What can you conclude from the plot?
\end{enumerate}
\end{exercise}

\begin{solution}[print=true]
  \begin{align*}
    \vec{c}^{\top} = (-2, -3), \quad
    \mat A =
    \begin{pmatrix}
      1 & 0 \\
      0  & 1\\
      2 & 1\\
      1 & 2
    \end{pmatrix}, \quad
    \vec{b}^{\top} = (2,3,6,7)
  \end{align*}
    \begin{equation*}
    \vec x^* = (1.67, 2.67)^{\top}
  \end{equation*}
\end{solution}


\begin{exercise}[subtitle={Notebook}]
\begin{enumerate}[label=\emph{\alph*)}]
\item Follow the instructions in the notebook and implement the method to find an admissible point given a set of constraints.
\item Verify your implementation by searching for an admissible point for the problem in Exercise 6.2 given the inadmissible point $\vec{x} = (100,100)$. Make sure your method really results in an admissible point.
%\item Apply your method to the example of the McDonald's diet (Exercise 5.4) with additional constraints for the maximum daily consumption of each nutrient. What can you conclude?
\end{enumerate}
\end{exercise}

\end{document}